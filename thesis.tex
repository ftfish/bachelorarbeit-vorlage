% Template for bachelor's/master's thesis for students at Karlsruhe Institute of Technology.
% Best for students majored in Computer Science or Mathematics.

% English version by Wei Zhou (ftfish@gmail.com)
% Based on the German version of Timo Bingmann


% This is just a template. Since there's no strict rule on how a bachelor's thesis must look like, every part of this template can be adapted.

\documentclass[12pt,a4paper,twoside]{scrartcl}

% Loading babel to enable automatic hypentation and multiple languages in the document.
% The last language in the option list will be used as default.
\usepackage[ngerman,english]{babel}

% Using T1 font encoding and the Latin Modern font
\usepackage[T1]{fontenc}
\usepackage{lmodern}

% Using utf8 as file encoding.
\usepackage[utf8]{inputenc}


% Page size. Using almost the whole A4 paper.
\usepackage[tmargin=22mm,bmargin=22mm,lmargin=20mm,rmargin=20mm]{geometry}


% Some standard packages for writing papers
\usepackage{latexsym,amsmath,amssymb,mathtools,textcomp}

% Paragraphs are not indented but there will be some space between paragraphs
\usepackage{parskip}

% For defining theorem-style environments like Lemma/Proof/Definition
\usepackage{amsthm}

% Fixing spacing problems before theorems due to \usepackage{parskip}
\begingroup
    \makeatletter
    \@for\theoremstyle:=definition,remark,plain\do{%
        \expandafter\g@addto@macro\csname th@\theoremstyle\endcsname{%
            \addtolength\thm@preskip\parskip
            }%
        }
\endgroup

% Some theorem-style environments
\newtheorem{theorem}{Theorem}[section]
\newtheorem{definition}[theorem]{Definition}
\newtheorem{lemma}[theorem]{Lemma}

% Setting equation numbers to <chapter>.<section>.<index>
\numberwithin{equation}{section}

% For inserting graphics into the document
\usepackage{graphicx}
\graphicspath{{images/}}

% For tables
\usepackage{array,multirow}

% Control layout of itemize, enumerate, description
\usepackage{enumitem}

\setlist[enumerate]{topsep=0pt}
\setlist[itemize]{topsep=0pt}
\setlist[description]{font=\normalfont,topsep=0pt}

\setlist[enumerate,1]{label=(\roman*)}

% TikZ for graphics in LaTeX
\usepackage{tikz}
\usetikzlibrary{calc}

% Have the current section and subsection in the header.
\usepackage{fancyhdr}

\fancypagestyle{plain}{
  \setlength\footskip{32pt}
  \fancyhead{}
  \fancyfoot{}
  \fancyfoot[LE,RO]{\normalsize\thepage}
  \renewcommand{\headrulewidth}{0pt}
  \renewcommand{\footrulewidth}{0pt}
}

\fancypagestyle{normal}{
  \setlength{\headheight}{20pt}
  \setlength\footskip{32pt}
  \fancyhead{}
  \fancyhead[LE]{\normalsize\textsc{\nouppercase{\leftmark}}}
  \fancyhead[RO]{\normalsize\textsc{\nouppercase{\rightmark}}}
  \fancyfoot{}
  \fancyfoot[LE,RO]{\normalsize\thepage}
  \renewcommand{\headrulewidth}{0.4pt}
  \renewcommand{\footrulewidth}{0pt}
}

% Hyperref for hyperlinks und cross refs
\usepackage{color}
\usepackage[pagebackref]{hyperref}
\usepackage[all]{hypcap}

\hypersetup{
  pdftitle={Title of the thesis}, %TODO
  pdfauthor={Author of the thesis}, %TODO
  pdfsubject={Keyword, more keywords}, %TODO
  colorlinks=true,
  pdfborder={0 0 0},
  bookmarksopen=true,
  bookmarksopenlevel=1,
  bookmarksnumbered=true,
  linkcolor=blue!60!black,
  %linkcolor=black,
  citecolor=blue!60!black,
  urlcolor=blue!60!black,
  filecolor=green!60!black,
  pdfpagemode=UseNone,
  unicode=true,
}

% Add the word "page" for pagebackref's in the bibliography.
\renewcommand*{\backreflastsep}{, }
\renewcommand*{\backreftwosep}{, }
\renewcommand*{\backref}[1]{}
\renewcommand*{\backrefalt}[4]{%
  \ifcase #1 %
No citations.% use \relax if you do not want the "No citations" message %TODO
  \or
(Page #2).%
  \else
(Pages #2).%
  \fi%
}


% For importing graphics from subdirectories.
\usepackage{import}

% Referencing figures, etc.
\newcommand{\reflst}[1]{\hyperref[#1]{Listing~\ref*{#1}}}
\newcommand{\refthm}[1]{\hyperref[#1]{Theorem~\ref*{#1}}}
\newcommand{\refdef}[1]{\hyperref[#1]{Definition~\ref*{#1}}}




% Package for inserting pseudo codes in the document.
\usepackage[ruled,vlined,linesnumbered,norelsize]{algorithm2e}
\DontPrintSemicolon
\def\NlSty#1{\textnormal{\fontsize{8}{10}\selectfont{}#1}}
\SetKwSty{texttt}
\SetCommentSty{emph}
\def\listalgorithmcfname{List of Algorithms}
\def\algorithmautorefname{Algorithm}
\let\chapter=\section % resolve a problem with algorithm2e

\begin{document}

%%%%%%%%%%%%%%%%%%%%%%%%%%%%%%%%%%%%%%%%%%%%%%%%%%%%%%%%%%%%%%%%%%%%%%

\pagestyle{empty} % no page number
\pagenumbering{alph}

% title page
\begin{titlepage}

  \begin{center}\large

    {\flushleft\includegraphics[height=17mm]{kit_logo_en.pdf} \hfill}
%    \includegraphics[height=20mm]{grouplogo-algo-blue.pdf}\quad\null

    \vfill

    Bachelor's Thesis
    \vspace*{2cm}

    {\bf\huge Title of the Bachelor's Thesis \\ and when it is longer. \par} %TODO
    % Be sure to fill in the field pdftitle={} above
    % mit \par am Ende stimmt der Zeilenabstand
  

    \vfill

    Name of the Author %TODO

    \vspace*{15mm}

    Handover Date: 19.10.2012 %TODO

    \vspace*{45mm}

    \begin{tabular}{rl}
      Supervisors: & Prof. Dr. Peter Sanders \\ %TODO
      & Dipl. Inform. Second Supervisor \\
    \end{tabular}
    
    \vspace*{10mm}

	% Deutsch
%    Institut für Theoretische Informatik, Algorithmik \\
%    Fakultät für Informatik \\
%    Karlsruher Institut für Technologie

    % English:
%     Institute of Theoretical Informatics, Algorithmics \\
     Department of Informatics \\
     Karlsruhe Institute of Technology

    \vspace*{12mm}
  \end{center}

\end{titlepage}

%%%%%%%%%%%%%%%%%%%%%%%%%%%%%%%%%%%%%%%%%%%%%%%%%%%%%%%%%%%%%%%%%%%%%%
\vspace*{0pt}\vfill

\selectlanguage{ngerman}
\hrule\medskip

Hiermit versichere ich, dass ich diese Arbeit selbständig verfasst und keine anderen, als die angegebenen Quellen und Hilfsmittel benutzt, die wörtlich oder inhaltlich übernommenen Stellen als solche kenntlich gemacht und die Satzung des Karlsruher Instituts für Technologie zur Sicherung guter wissenschaftlicher Praxis in der jeweils gültigen Fassung beachtet habe.

\bigskip

\noindent
Ort, den Datum %TODO

% Hand-written signature!! %TODO

\vspace*{5cm}

\clearpage

%%%%%%%%%%%%%%%%%%%%%%%%%%%%%%%%%%%%%%%%%%%%%%%%%%%%%%%%%%%%%%%%%%%%%%

\vspace*{0pt}\vfill

\selectlanguage{ngerman}
\begin{abstract}
\centerline{\bf Zusammenfassung}

Hier die deutsche Zusammenfassung.

Ich bin Blindtext. Von Geburt an. Es hat lange gedauert, bis ich begriffen habe, was es bedeutet, ein blinder Text zu sein: Man macht keinen Sinn. Man wirkt hier und da aus dem Zusammenhang gerissen. Oft wird man gar nicht erst gelesen. Aber bin ich deshalb ein schlechter Text? Ich weiß, dass ich nie die Chance haben werde im Stern zu erscheinen. Aber bin ich darum weniger wichtig? Ich bin blind! Aber ich bin gerne Text. Und sollten Sie mich jetzt tatsächlich zu Ende lesen, dann habe ich etwas geschafft, was den meisten ,,normalen`` Texten nicht gelingt.

Ich bin Blindtext. Von Geburt an. Es hat lange gedauert, bis ich begriffen habe, was es bedeutet, ein blinder Text zu sein: Man macht keinen Sinn. Man wirkt hier und da aus dem Zusammenhang gerissen. Oft wird man gar nicht erst gelesen. Aber bin ich deshalb ein schlechter Text? Ich weiß, dass ich nie die Chance haben werde im Stern zu erscheinen. Aber bin ich darum weniger wichtig? Ich bin blind! Aber ich bin gerne Text.

\end{abstract}

\vfill

\selectlanguage{english}
\begin{abstract}
\centerline{\bf Abstract}

And here an English translation of the German abstract.

I'm blind text. From birth. It took a long time until I realized what it means to be random text: You make no sense. You stand here and there out of context. Frequently, they do not even read. But I have a bad copy? I know that I will never have the chance of appearing in the. But I'm any less important? I'm blind! But I like to text. And you should see me now actually over, then I have accomplished something that is not possible in most ``normal'' copies.

I'm blind text. From birth. It took a long time until I realized what it means to be random text: You make no sense. You stand here and there out of context. Frequently, they do not even read. But I have a bad copy? I know that I will never have the chance of appearing in the. But I'm any less important? I'm blind! But I like to text.

\end{abstract}

\vfill\vfill\vfill
\clearpage

%%%%%%%%%%%%%%%%%%%%%%%%%%%%%%%%%%%%%%%%%%%%%%%%%%%%%%%%%%%%%%%%%%%%%%

\vspace*{0pt}\vfill

\section*{Acknowledgments}

Lorem ipsum dolor sit amet, consectetuer adipiscing elit. Aenean commodo ligula eget dolor. Aenean massa. Cum sociis natoque penatibus et magnis dis parturient montes, nascetur ridiculus mus. Donec quam felis, ultricies nec, pellentesque eu, pretium quis, sem. Nulla consequat massa quis enim. Donec pede justo, fringilla vel, aliquet nec, vulputate eget, arcu. In enim justo, rhoncus ut, imperdiet a, venenatis vitae, justo.

\vfill\vfill\vfill
\clearpage

%%%%%%%%%%%%%%%%%%%%%%%%%%%%%%%%%%%%%%%%%%%%%%%%%%%%%%%%%%%%%%%%%%%%%%
\selectlanguage{english}
\pagestyle{plain}
\pagenumbering{roman}
  
% markiere sections im Seitenkopf links und subsections rechts
\renewcommand\sectionmark[1]{\markboth{\thesection\quad\MakeUppercase{#1}}{\thesection\quad\MakeUppercase{#1}}}
\renewcommand\subsectionmark[1]{\markright{\thesubsection\quad\MakeUppercase{#1}}}


\tableofcontents

\clearpage

%%%%%%%%%%%%%%%%%%%%%%%%%%%%%%%%%%%%%%%%%%%%%%%%%%%%%%%%%%%%%%%%%%%%%%

\listoffigures
\listoftables
\listofalgorithms

\clearpage

%%%%%%%%%%%%%%%%%%%%%%%%%%%%%%%%%%%%%%%%%%%%%%%%%%%%%%%%%%%%%%%%%%%%%%
\pagestyle{normal}
\pagenumbering{arabic}

\section{Introduction}

\subsection{Motivation}

Weit hinten, hinter den Wortbergen, fern der Länder Vokalien und Konsonantien leben die Blindtexte. Abgeschieden wohnen Sie in Buchstabhausen an der Küste des Semantik, eines großen Sprachozeans. Ein kleines Bächlein namens Duden fließt durch ihren Ort und versorgt sie mit den nötigen Regelialien. Es ist ein paradiesmatisches Land, in dem einem gebratene Satzteile in den Mund fliegen. Nicht einmal von der allmächtigen Interpunktion werden die Blindtexte beherrscht - ein geradezu unorthographisches Leben. Eines Tages aber beschloß eine kleine Zeile Blindtext, ihr Name war Lorem Ipsum, hinaus zu gehen in die weite Grammatik. Der große Oxmox riet ihr davon ab, da es dort wimmele von bösen Kommata, wilden Fragezeichen und hinterhältigen Semikoli, doch das Blindtextchen ließ sich nicht beirren. Es packte seine sieben Versalien, schob sich sein Initial in den Gürtel und machte sich auf den Weg. Als es die ersten Hügel des Kursivgebirges erklommen hatte, warf es einen letzten Blick zurück auf die Skyline seiner Heimatstadt Buchstabhausen, die Headline von Alphabetdorf und die Subline seiner eigenen Straße, der Zeilengasse.

Wehmütig lief ihm eine rhetorische Frage über die Wange, dann setzte es seinen Weg fort. Unterwegs traf es eine Copy. Die Copy warnte das Blindtextchen, da, wo sie herkäme wäre sie zigmal umgeschrieben worden und alles, was von ihrem Ursprung noch übrig wäre, sei das Wort ,,und`` und das Blindtextchen solle umkehren und wieder in sein eigenes, sicheres Land zurückkehren. Doch alles Gutzureden konnte es nicht überzeugen und so dauerte es nicht lange, bis ihm ein paar heimtückische Werbetexter auflauerten, es mit Longe und Parole betrunken machten und es dann in ihre Agentur schleppten, wo sie es für ihre Projekte wieder und wieder mißbrauchten. Und wenn es nicht umgeschrieben wurde, dann benutzen Sie es immernoch.
\begin{enumerate}
\item Weit hinten, hinter den Wortbergen, fern der Länder Vokalien und Konsonantien leben die Blindtexte.
\item Abgeschieden wohnen Sie in Buchstabhausen an der Küste des Semantik, eines großen Sprachozeans.
\item
Ein kleines Bächlein namens Duden fließt durch ihren Ort und versorgt sie mit den nötigen Regelialien. Es ist ein paradiesmatisches Land, in dem einem gebratene Satzteile in den Mund fliegen.
\end{enumerate}
Nicht einmal von der allmächtigen Interpunktion werden die Blindtexte beherrscht - ein geradezu unorthographisches Leben. Eines Tages aber beschloß eine kleine Zeile Blindtext, ihr Name war Lorem Ipsum, hinaus zu gehen in die weite Grammatik. Der große Oxmox riet ihr davon ab, da es dort wimmele von bösen Kommata, wilden Fragezeichen und hinterhältigen Semikoli, doch das Blindtextchen ließ sich nicht beirren. Es packte seine sieben Versalien, schob sich sein Initial in den Gürtel und machte sich auf den Weg. Als es die ersten Hügel des Kursivgebirges erklommen hatte, warf es einen letzten Blick zurück auf die Skyline seiner Heimatstadt Buchstabhausen, die Headline von Alphabetdorf und die Subline seiner eigenen Straße, der Zeilengasse. Wehmütig lief ihm eine rhetorische Frage über die Wange, dann setzte es seinen Weg fort. Unterwegs traf es eine Copy. Die Copy warnte das Blindtextchen, da, wo sie herkäme wäre sie zigmal umgeschrieben worden und alles, was von ihrem Ursprung noch übrig wäre, sei das Wort ,,und`` und das Blindtextchen solle umkehren und wieder in sein eigenes, sicheres Land zurückkehren.

\subsection{Contribution of the Thesis}

Doch alles Gutzureden konnte es nicht überzeugen und so dauerte es nicht lange, bis ihm ein paar heimtückische Werbetexter auflauerten, es mit Longe und Parole betrunken machten und es dann in ihre Agentur schleppten, wo sie es für ihre Projekte wieder und wieder mißbrauchten. Und wenn es nicht umgeschrieben wurde, dann benutzen Sie es immernoch. Weit hinten, hinter den Wortbergen, fern der Länder Vokalien und Konsonantien leben die Blindtexte. Abgeschieden wohnen Sie in Buchstabhausen an der Küste des Semantik, eines großen Sprachozeans. Ein kleines Bächlein namens Duden fließt durch ihren Ort und versorgt sie mit den nötigen Regelialien. Es ist ein paradiesmatisches Land, in dem einem gebratene Satzteile in den Mund fliegen. Nicht einmal von der allmächtigen Interpunktion werden die Blindtexte beherrscht - ein geradezu unorthographisches Leben. Eines Tages aber beschloß eine kleine Zeile Blindtext, ihr Name war Lorem Ipsum, hinaus zu gehen in die weite Grammatik. Der große Oxmox riet ihr davon ab, da es dort wimmele von bösen Kommata, wilden Fragezeichen und hinterhältigen Semikoli, doch das Blindtextchen ließ sich nicht beirren.

%%%%%%%%%%%%%%%%%%%%%%%%%%%%%%%%%%%%%%%%%%%%%%%%%%%%%%%%%%%%%%%%%%%%%%

\section{Known Algorithms}

Es packte seine sieben Versalien \cite{Cormen2001introduction}, schob sich sein Initial in den Gürtel und machte sich auf den Weg. Als es die ersten Hügel des Kursivgebirges erklommen hatte, warf es einen letzten Blick zurück auf die Skyline seiner Heimatstadt Buchstabhausen, die Headline von Alphabetdorf und die Subline seiner eigenen Straße, der Zeilengasse. Wehmütig lief ihm eine rhetorische Frage über die Wange, dann \cite{Cormen2001introduction} setzte es seinen Weg fort. Unterwegs traf es eine Copy.
\begin{definition}\label{def:blind}
Die Copy warnte das Blindtextchen, da, wo sie herkäme wäre sie zigmal umgeschrieben worden und alles, was von ihrem Ursprung noch übrig wäre, sei das Wort ,,und`` und das Blindtextchen solle umkehren und wieder in sein eigenes, sicheres Land zurückkehren.
\end{definition}
Doch alles Gutzureden konnte es nicht überzeugen und so dauerte es nicht lange, bis ihm ein paar heimtückische Werbetexter auflauerten, es mit Longe und Parole betrunken machten und es dann in ihre Agentur schleppten, wo sie es für ihre Projekte wieder und wieder mißbrauchten. Und wenn es nicht umgeschrieben wurde, dann benutzen Sie es immernoch. Weit hinten, hinter den Wortbergen, fern der Länder Vokalien und Konsonantien leben die Blindtexte.

\subsection{Enumeration of All Possibilities}

\begin{algorithm}
  \caption{Tiefensuche -- depth-first search}\label{alg:Tiefensuche}
  \SetKwFunction{DFS}{DFS}

  \KwIn{Ein Graph $G = (V,E)$}

  \While(\tcp*[f]{Lösche Knotennummer $\nu[v]$, Vorgänger $\pi[v]$}){$v \in V$}{
    $\nu[v] \leftarrow nil$, $\pi[v] \leftarrow nil$, $C[v] \leftarrow nil$ \tcp*{und Komponente $C[v]$.}
  }
  $\nu \leftarrow 0$, $C \leftarrow 0$ \tcp*{Initiale Knoten- und Komponentennummern.}

  \DFS{$v$}
  \Begin{
    $\nu[v] \leftarrow \nu$, $\nu \leftarrow \nu+1$, $C[v] \leftarrow C$ \tcp*{Nummeriere Knoten.}
    \ForEach{$w \in \operatorname{Adj}(v) \KwSty{ mit } \nu[w] = nil$}{
      $\pi[w] \leftarrow v$, \DFS{$w$} \tcp*{Rekursion auf unnummerierte Nachbarn.}
    }
  }

  \ForEach{$r \in V$}{
    \lIf(\tcp*{Beginne bei Wurzelknoten $r$.}){$\nu[r] = nil$}{
      \DFS{$r$}, $C \leftarrow C+1$
    }
  }

  \KwOut{$(V,\{ \{v,\pi[v]\} \mid v \in V ,\, \pi[v] \neq nil \})$ ist ein $G$ aufspannender Wald und $\{ \{ v \mid v \in V \text{ und } C[v] = C \} \mid C \in \mathbb{N}_0 \}$ sind die Komponenten von $G$.}

\end{algorithm}

Abgeschieden wohnen Sie in Buchstabhausen an der Küste des Semantik, eines großen Sprachozeans. Ein kleines Bächlein namens Duden fließt durch ihren Ort und versorgt sie mit den nötigen Regelialien. Es ist ein paradiesmatisches Land, in dem einem gebratene Satzteile in den Mund fliegen (siehe \autoref{alg:Tiefensuche}). Nicht einmal von der allmächtigen Interpunktion werden die Blindtexte beherrscht - ein geradezu unorthographisches Leben. Eines Tages aber beschloß eine kleine Zeile Blindtext, ihr Name war Lorem Ipsum, hinaus zu gehen in die weite Grammatik.

\subsection{Reformulating as a Linear Program}

Der große Oxmox riet ihr davon ab, da es dort wimmele von bösen Kommata, wilden Fragezeichen und hinterhältigen Semikoli, doch das Blindtextchen ließ sich nicht beirren. Es packte seine sieben Versalien, schob sich sein Initial in den Gürtel und machte sich auf den Weg. Als es die ersten Hügel des Kursivgebirges erklommen hatte, warf es einen letzten Blick zurück auf die Skyline seiner Heimatstadt Buchstabhausen, die Headline von Alphabetdorf und die Subline seiner eigenen Straße, der Zeilengasse. Wehmütig lief ihm eine rhetorische Frage über die Wange, dann setzte es seinen Weg fort. Unterwegs traf es eine Copy:
$$\max \{ c x \mid x \in \mathbb{R}^n \,,\; x \geq 0 \,,\; A x \leq b \} \,.$$
Die Copy warnte das Blindtextchen, da, wo sie herkäme wäre sie zigmal umgeschrieben worden und alles, was von ihrem Ursprung noch übrig wäre, sei das Wort ,,und`` und das Blindtextchen solle umkehren und wieder in sein eigenes, sicheres Land zurückkehren. Doch alles Gutzureden konnte es nicht überzeugen und so dauerte es nicht lange, bis ihm ein paar heimtückische Werbetexter auflauerten, es mit Longe und Parole betrunken machten und es dann in ihre Agentur schleppten, wo sie es für ihre Projekte wieder und wieder mißbrauchten. Und wenn es nicht umgeschrieben wurde, dann benutzen Sie es immernoch.
\begin{theorem}\label{thm:blind}
Weit hinten, hinter den Wortbergen, fern der Länder Vokalien und Konsonantien leben die Blindtexte.
\begin{proof}
Abgeschieden wohnen Sie in Buchstabhausen an der Küste des Semantik, eines großen Sprachozeans. Ein kleines Bächlein namens Duden fließt durch ihren Ort und versorgt sie mit den nötigen Regelialien. Es ist ein paradiesmatisches Land, in dem einem gebratene Satzteile in den Mund fliegen. Nicht einmal von der allmächtigen Interpunktion werden die Blindtexte beherrscht - ein geradezu unorthographisches Leben.
\end{proof}
\end{theorem}

%%%%%%%%%%%%%%%%%%%%%%%%%%%%%%%%%%%%%%%%%%%%%%%%%%%%%%%%%%%%%%%%%%%%%%

\section{Implementation}

Eines Tages \cite{Cormen2001introduction} aber beschloß eine kleine Zeile Blindtext, ihr Name war Lorem Ipsum, hinaus zu gehen in die weite Grammatik. Der große Oxmox riet ihr davon ab, da es dort wimmele von bösen Kommata, wilden Fragezeichen und hinterhältigen Semikoli, doch das Blindtextchen ließ sich nicht beirren. Es packte seine sieben Versalien, schob sich sein Initial in den Gürtel und machte sich auf den Weg. Als es die ersten Hügel des Kursivgebirges erklommen hatte, warf es einen letzten Blick zurück auf die Skyline seiner Heimatstadt Buchstabhausen, die Headline von Alphabetdorf und die Subline seiner eigenen Straße, der Zeilengasse. Wehmütig lief ihm eine rhetorische Frage über die Wange, dann setzte es seinen Weg fort. Unterwegs traf es eine Copy. Die Copy warnte das Blindtextchen, da, wo sie herkäme wäre sie zigmal umgeschrieben worden und alles, was von ihrem Ursprung noch übrig wäre, sei das Wort ,,und`` und das Blindtextchen solle umkehren und wieder in sein eigenes, sicheres Land zurückkehren.

Doch alles Gutzureden konnte es nicht überzeugen und so dauerte es nicht lange, bis ihm ein paar heimtückische Werbetexter auflauerten, es mit Longe und Parole betrunken machten und es dann in ihre Agentur schleppten, wo sie es für ihre Projekte wieder und wieder mißbrauchten. Und wenn es nicht umgeschrieben wurde, dann benutzen Sie es immernoch. Weit hinten, hinter den Wortbergen, fern der Länder Vokalien und Konsonantien leben die Blindtexte. Abgeschieden wohnen Sie in Buchstabhausen an der Küste des Semantik, eines großen Sprachozeans. Ein kleines Bächlein namens Duden fließt durch ihren Ort und versorgt sie mit den nötigen Regelialien. Es ist ein paradiesmatisches Land, in dem einem gebratene Satzteile in den Mund fliegen (siehe \autoref{fig:binäre Suchbäume}). Nicht einmal von der allmächtigen Interpunktion werden die Blindtexte beherrscht - ein geradezu unorthographisches Leben. Eines Tages aber beschloß eine kleine Zeile Blindtext, ihr Name war Lorem Ipsum, hinaus zu gehen in die weite Grammatik. Der große Oxmox riet ihr davon ab, da es dort wimmele von bösen Kommata, wilden Fragezeichen und hinterhältigen Semikoli, doch das Blindtextchen ließ sich nicht beirren.

\begin{figure}
\hfill%
\begin{tikzpicture}[font=\small,
  odot/.style={draw,circle,inner sep=0pt,minimum width=18pt},
  ]
  \node[odot] (0) at (0, -0) {25};
  \node[odot] (1) at (-2, -1) {10};
  \node[odot] (2) at (2, -1) {31};
  \node[odot] (3) at (-3, -2) {4};
  \node[odot] (4) at (-1, -2) {21};
  \node[odot] (5) at (1, -2) {30};
  \node[odot] (6) at (3, -2) {32};
  \node[odot] (7) at (-3.5, -3) {2};
  \node[odot] (8) at (-2.5, -3) {8};
  \node[odot] (9) at (-1.5, -3) {11};
  \node[odot] (10) at (-0.5, -3) {24};
  \node[odot] (11) at (0.5, -3) {28};
  \draw (4) edge (10);
  \draw (3) edge (8);
  \draw (1) edge (3);
  \draw (0) edge (2);
  \draw (2) edge (6);
  \draw (4) edge (9);
  \draw (2) edge (5);
  \draw (1) edge (4);
  \draw (0) edge (1);
  \draw (3) edge (7);
  \draw (5) edge (11);
\end{tikzpicture}
\hfill%
\begin{tikzpicture}[font=\small,
  odot/.style={draw,circle,inner sep=0pt,minimum width=18pt},
  ]
  \node[odot] (0) at (0, -0) {21};
  \node[odot] (1) at (-2, -1) {8};
  \node[odot] (2) at (2, -1) {30};
  \node[odot] (3) at (-3, -2) {4};
  \node[odot] (4) at (-1, -2) {11};
  \node[odot] (5) at (1, -2) {25};
  \node[odot] (6) at (3, -2) {31};
  \node[odot] (7) at (-3.5, -3) {2};
  \node[odot] (8) at (-1.5, -3) {10};
  \node[odot] (9) at (0.5, -3) {24};
  \node[odot] (10) at (1.5, -3) {28};
  \node[odot] (11) at (3.5, -3) {32};
  \draw (5) edge (9);
  \draw (0) edge (2);
  \draw (1) edge (4);
  \draw (5) edge (10);
  \draw (2) edge (5);
  \draw (6) edge (11);
  \draw (0) edge (1);
  \draw (4) edge (8);
  \draw (3) edge (7);
  \draw (1) edge (3);
  \draw (2) edge (6);
\end{tikzpicture}
\hfill\null%

\caption{Zwei binäre Suchbäume für $\{ 2, 4, 8, 10, 11, 21, 24, 25, 28, 30, 31, 32 \}$}\label{fig:binäre Suchbäume}
\end{figure}

Es packte seine sieben Versalien, schob sich sein Initial in den Gürtel und machte sich auf den Weg. Als es die ersten Hügel des Kursivgebirges erklommen hatte, warf es einen letzten Blick zurück auf die Skyline seiner Heimatstadt Buchstabhausen, die Headline von Alphabetdorf und die Subline seiner eigenen Straße, der Zeilengasse. Wehmütig lief ihm eine rhetorische Frage über die Wange, dann setzte es seinen Weg fort. Unterwegs traf es eine Copy. Die Copy warnte das Blindtextchen, da, wo sie herkäme wäre sie zigmal umgeschrieben worden und alles, was von ihrem Ursprung noch übrig wäre, sei das Wort ,,und`` und das Blindtextchen solle umkehren und wieder in sein eigenes, sicheres Land zurückkehren. Doch alles Gutzureden konnte es nicht überzeugen und so dauerte es nicht lange, bis ihm ein paar heimtückische Werbetexter auflauerten, es mit Longe und Parole betrunken machten und es dann in ihre Agentur schleppten, wo sie es für ihre Projekte wieder und wieder mißbrauchten. Und wenn es nicht umgeschrieben wurde, dann benutzen Sie es immernoch. Weit hinten, hinter den Wortbergen, fern der Länder Vokalien und Konsonantien leben die Blindtexte.

Abgeschieden wohnen Sie in Buchstabhausen an der Küste des Semantik, eines großen Sprachozeans. Ein kleines Bächlein namens Duden fließt durch ihren Ort und versorgt sie mit den nötigen Regelialien. Es ist ein paradiesmatisches Land, in dem einem gebratene Satzteile in den Mund fliegen. Nicht einmal von der allmächtigen Interpunktion werden die Blindtexte beherrscht - ein geradezu unorthographisches Leben. Eines Tages aber beschloß eine kleine Zeile Blindtext, ihr Name war Lorem Ipsum, hinaus zu gehen in die weite Grammatik. Der große Oxmox riet ihr davon ab, da es dort wimmele von bösen Kommata, wilden Fragezeichen und hinterhältigen Semikoli, doch das Blindtextchen ließ sich nicht beirren. Es packte seine sieben Versalien, schob sich sein Initial in den Gürtel und machte sich auf den Weg. Als es die ersten Hügel des Kursivgebirges erklommen hatte, warf es einen letzten Blick zurück auf die Skyline seiner Heimatstadt Buchstabhausen, die Headline von Alphabetdorf und die Subline seiner eigenen Straße, der Zeilengasse. Wehmütig lief ihm eine rhetorische Frage über die Wange, dann setzte es seinen Weg fort. Unterwegs traf es eine Copy.

%%%%%%%%%%%%%%%%%%%%%%%%%%%%%%%%%%%%%%%%%%%%%%%%%%%%%%%%%%%%%%%%%%%%%%

\section{Experimental Results}

\subsection{Running Time of the Enumeration}

Die Copy warnte das Blindtextchen, da, wo sie herkäme wäre sie zigmal umgeschrieben worden und alles, was von ihrem Ursprung noch übrig wäre, sei das Wort ,,und`` und das Blindtextchen solle umkehren und wieder in sein eigenes, sicheres Land zurückkehren. Doch alles Gutzureden konnte es nicht überzeugen und so dauerte es nicht lange, bis ihm ein paar heimtückische Werbetexter auflauerten, es mit Longe und Parole betrunken machten und es dann in ihre Agentur schleppten, wo sie es für ihre Projekte wieder und wieder mißbrauchten. Und wenn es nicht umgeschrieben wurde, dann benutzen Sie es immernoch. Weit hinten, hinter den Wortbergen, fern der Länder Vokalien und Konsonantien leben die Blindtexte. Abgeschieden wohnen Sie in Buchstabhausen an der Küste des Semantik, eines großen Sprachozeans. Ein kleines Bächlein namens Duden fließt durch ihren Ort und versorgt sie mit den nötigen Regelialien. Es ist ein paradiesmatisches Land, in dem einem gebratene Satzteile in den Mund fliegen. Nicht einmal von der allmächtigen Interpunktion werden die Blindtexte beherrscht - ein geradezu unorthographisches Leben.

Eines Tages aber beschloß eine kleine Zeile Blindtext, ihr Name war Lorem Ipsum, hinaus zu gehen in die weite Grammatik. Der große Oxmox riet ihr davon ab, da es dort wimmele von bösen Kommata, wilden Fragezeichen und hinterhältigen Semikoli, doch das Blindtextchen ließ sich nicht beirren. Es packte seine sieben Versalien, schob sich sein Initial in den Gürtel und machte sich auf den Weg. Als es die ersten Hügel des Kursivgebirges erklommen hatte, warf es einen letzten Blick zurück auf die Skyline seiner Heimatstadt Buchstabhausen, die Headline von Alphabetdorf und die Subline seiner eigenen Straße, der Zeilengasse. Wehmütig lief ihm eine rhetorische Frage über die Wange, dann setzte es seinen Weg fort. Unterwegs traf es eine Copy. Die Copy warnte das Blindtextchen, da, wo sie herkäme wäre sie zigmal umgeschrieben worden und alles, was von ihrem Ursprung noch übrig wäre, sei das Wort ,,und`` und das Blindtextchen solle umkehren und wieder in sein eigenes, sicheres Land zurückkehren.

\subsection{Running time with CPLEX}

Doch alles Gutzureden konnte es nicht überzeugen und so dauerte es nicht lange, bis ihm ein paar heimtückische Werbetexter auflauerten, es mit Longe und Parole betrunken machten und es dann in ihre Agentur schleppten, wo sie es für ihre Projekte wieder und wieder mißbrauchten. Und wenn es nicht umgeschrieben wurde, dann benutzen Sie es immernoch. Weit hinten, hinter den Wortbergen, fern der Länder Vokalien und Konsonantien leben die Blindtexte. Abgeschieden wohnen Sie in Buchstabhausen an der Küste des Semantik, eines großen Sprachozeans. Ein kleines Bächlein namens Duden fließt durch ihren Ort und versorgt sie mit den nötigen Regelialien. Es ist ein paradiesmatisches Land, in dem einem gebratene Satzteile in den Mund fliegen. Nicht einmal von der allmächtigen Interpunktion werden die Blindtexte beherrscht - ein geradezu unorthographisches Leben. Eines Tages aber beschloß eine kleine Zeile Blindtext, ihr Name war Lorem Ipsum, hinaus zu gehen in die weite Grammatik. Der große Oxmox riet ihr davon ab, da es dort wimmele von bösen Kommata, wilden Fragezeichen und hinterhältigen Semikoli, doch das Blindtextchen ließ sich nicht beirren.

I can't help inserting a Gnuplot graphics here. See \autoref{fig:pm3d17}.

\begin{figure}
\centering
\subimport{images/}{pm3d17}
\caption{A plot generated with gnuplot}
\label{fig:pm3d17}
\end{figure}

%%%%%%%%%%%%%%%%%%%%%%%%%%%%%%%%%%%%%%%%%%%%%%%%%%%%%%%%%%%%%%%%%%%%%%

\section{Conclusion}

\subsection{Results and Evaluation}

Es packte seine sieben Versalien, schob sich sein Initial in den Gürtel und machte sich auf den Weg. Als es die ersten Hügel des Kursivgebirges erklommen hatte, warf es einen letzten Blick zurück auf die Skyline seiner Heimatstadt Buchstabhausen, die Headline von Alphabetdorf und die Subline seiner eigenen Straße, der Zeilengasse. Wehmütig lief ihm eine rhetorische Frage über die Wange, dann setzte es seinen Weg fort. Unterwegs traf es eine Copy. Die Copy warnte das Blindtextchen, da, wo sie herkäme wäre sie zigmal umgeschrieben worden und alles, was von ihrem Ursprung noch übrig wäre, sei das Wort ,,und`` und das Blindtextchen solle umkehren und wieder in sein eigenes, sicheres Land zurückkehren. Doch alles Gutzureden konnte es nicht überzeugen und so dauerte es nicht lange, bis ihm ein paar heimtückische Werbetexter auflauerten, es mit Longe und Parole betrunken machten und es dann in ihre Agentur schleppten, wo sie es für ihre Projekte wieder und wieder mißbrauchten. Und wenn es nicht umgeschrieben wurde, dann benutzen Sie es immernoch. Weit hinten, hinter den Wortbergen, fern der Länder Vokalien und Konsonantien leben die Blindtexte.

Abgeschieden wohnen Sie in Buchstabhausen an der Küste des Semantik, eines großen Sprachozeans. Ein kleines Bächlein namens Duden fließt durch ihren Ort und versorgt sie mit den nötigen Regelialien, see \refthm{thm:blind}. Es ist ein paradiesmatisches Land, in dem einem gebratene Satzteile in den Mund fliegen. Nicht einmal von der allmächtigen Interpunktion werden die Blindtexte beherrscht - ein geradezu unorthographisches Leben. Eines Tages aber beschloß eine kleine Zeile Blindtext, ihr Name war Lorem Ipsum, hinaus zu gehen in die weite Grammatik. Der große Oxmox riet ihr davon ab, da es dort wimmele von bösen Kommata, wilden Fragezeichen und hinterhältigen Semikoli, doch das Blindtextchen ließ sich nicht beirren. Es packte seine sieben Versalien, schob sich sein Initial in den Gürtel und machte sich auf den Weg. Als es die ersten Hügel des Kursivgebirges erklommen hatte, warf es einen letzten Blick zurück auf die Skyline seiner Heimatstadt Buchstabhausen, die Headline von Alphabetdorf und die Subline seiner eigenen Straße, der Zeilengasse. Wehmütig lief ihm eine rhetorische Frage über die Wange, dann setzte es seinen Weg fort. Unterwegs traf es eine Copy.

\subsection{Possible Extensions}

Die Copy warnte das Blindtextchen, da, wo sie herkäme wäre sie zigmal umgeschrieben worden und alles, was von ihrem Ursprung noch übrig wäre, sei das Wort ,,und`` und das Blindtextchen solle umkehren und wieder in sein eigenes, sicheres Land zurückkehren. Doch alles Gutzureden konnte es nicht überzeugen und so dauerte es nicht lange, bis ihm ein paar heimtückische Werbetexter auflauerten, es mit Longe und Parole betrunken machten und es dann in ihre Agentur schleppten, wo sie es für ihre Projekte wieder und wieder mißbrauchten. Und wenn es nicht umgeschrieben wurde, dann benutzen Sie es immernoch. Weit hinten, hinter den Wortbergen, fern der Länder Vokalien und Konsonantien leben die Blindtexte. Abgeschieden wohnen Sie in Buchstabhausen an der Küste des Semantik, eines großen Sprachozeans. Ein kleines Bächlein namens Duden fließt durch ihren Ort und versorgt sie mit den nötigen Regelialien. Es ist ein paradiesmatisches Land, in dem einem gebratene Satzteile in den Mund fliegen. Nicht einmal von der allmächtigen Interpunktion werden die Blindtexte beherrscht - ein geradezu unorthographisches Leben. See \refdef{def:blind} for more information.

\clearpage

%%%%%%%%%%%%%%%%%%%%%%%%%%%%%%%%%%%%%%%%%%%%%%%%%%%%%%%%%%%%%%%%%%%%%%

\bibliographystyle{ieeetr}
\bibliography{references}

\end{document}
